\documentclass[a4paper,10pt]{article}
\usepackage[utf8x]{inputenc}
\usepackage{amsmath}
\usepackage{amssymb}

\begin{document}
\begin{center}


Arpon Basu,School: AECS-4,Mumbai-400094

Solution for problem 511
\end{center}

   We claim that the \textbf{minima is} $\mathbf{94}$ which is attained when $$(x_{1},x_{2}, \cdots ,x_{40})=(\underbrace{1,1, \cdots 1,1}_{22},\underbrace{2,2 \cdots 2,2}_{18})$$
   If the minima occurs for some other tuplet where $r$ twos have become one and $k$ ones have increased , like 
   $$ (\underbrace{1,1, \cdots 1,1}_{k},\underbrace{2,2 \cdots 2,2}_{r}) \longrightarrow (1+x_{1},1+x_{2}, \cdots 1+x_{k},\underbrace{1,1 \cdots 1,1}_{r})$$ 
   where $x_{i}>0$ and $\sum_{i=1}^{k} x_{i} = r $ then: \\
  
   Original sum ($S_{1} $) $$S_{1} = k\cdot 1^2 + r\cdot 2^2 = 4r+k$$ \\
   \\
   Changed Sum$(S_{2})$   $$ S_{2}=\sum_{i=1}^{k} (1+x_{i})^2 + r\cdot 1^2=k+2 \sum_{i=1}^{k} x_{i} +\sum_{i=1}^{k} x_{i}^2 + r $$\\
   \\
   
   
 
   

\begin{equation}
\begin{split}
\therefore S_{2}-S_{1} & =k+3r+ (\sum_{i=1}^{k} x_{i}^2) -4r - k \\
& = (\sum_{i=1}^{k} x_{i}^2) -r \\
& =(\sum_{i=1}^{k} x_{i}^2) - (\sum_{i=1}^{k} x_{i}) \hspace{8pt}\text{(replacing $r$ by $\sum_{i=1}^{k} x_{i}$)}\\
& \ge 0
\end{split}
\end{equation}




   
   
   
   \begin{center}
      \fbox{$\therefore 94$ is indeed the minima.}\\
    \end{center}
   
   We claim that \textbf{maxima is} $\mathbf{400}$ which is attained when $$(x_{1},x_{2}, \cdots x_{40})=(\underbrace{1,1, \cdots 1,1}_{39},19)$$ 
   If any other tuplet achieves maxima like
   
    $$ (1+x_{1},1+x_{2}, \cdots 1+x_{k},19-r)$$ where $\sum_{i=1}^{k} x_{i} = r $ and $r \le 18$ then: \\
    
    Original sum($S_{1}$) $$S_{1} = k\cdot 1^2 + 19^2 = k+361$$ \\
    
    Changed  Sum($S_{2}$) % $$S_{2}=\sum_{i=1}^{k} (1+x_{i})^2 + (19-r)^2=k+ \sum_{i=1}^{k} x_{i}^2 + 2 \sum_{i=1}^{k} x_{i} + 361 + r^2 -38r$$
   
   % $\therefore  S_{2}-S_{1}=361+k+ \sum_{i=1}^{k} x_{i}^2 +r^2 -36r -361-k = \sum_{i=1}^{k} x_{i}^2 +r^2 -36r = \sum_{i=1}^{k} x_{i}^2 +(\sum_{i=1}^{k} x_{i})^2 - 36r < 2(\sum_{i=1}^{k} x_{i})^2 -36r=2r(r-18) \le 0$ \\

\begin{equation}
\begin{split}
 S_{2} & =\sum_{i=1}^{k} (1+x_{i})^2 + (19-r)^2 \\
       & = k+ \sum_{i=1}^{k} x_{i}^2 + 2 \sum_{i=1}^{k} x_{i} + 361 + r^2 -38r \\
       & = 361+k + \sum_{i=1}^{k} x_{i}^2 + 2r + r^2 -38r  \hspace{8pt}\text{(replacing $\sum_{i=1}^{k}$ by $r$)}  \\
       & = 361+k + \sum_{i=1}^{k} x_{i}^2 + r^2 -36r \\
\end{split}
\end{equation}


   
\begin{equation}
\begin{split}
  \therefore  S_{2}-S_{1} & = 361+k+ \sum_{i=1}^{k} x_{i}^2 +r^2 -36r -361-k \\
                          & = \sum_{i=1}^{k} x_{i}^2 +r^2 -36r \\
                          & = \sum_{i=1}^{k} x_{i}^2 +(\sum_{i=1}^{k} x_{i})^2 - 36r \\
                          & < 2(\sum_{i=1}^{k} x_{i})^2 -36r \\
                          & =2r^2-36r \\
                          & =2r(r-18) \\      
                          &  \le 0
\end{split}
\end{equation}




    \begin{center}
      \fbox{$\therefore 400$ is indeed the maxima.}\\
    \end{center}

    
    
    
\end{document}

    
 

