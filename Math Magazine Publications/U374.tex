\documentclass[10pt,a4paper]{extarticle}
\usepackage[a4paper,margin=6mm]{geometry}
\usepackage{amsmath}


\begin{document}
 
Arpon Basu, School: AECS-4 , Mumbai-400094\\
 
Solution for Problem U374 \\

Without loss of generality,we can assume that the two equal roots are a and b, thus $a=b$\\
Therefore the Viete's relations give that:-\\

\begin{equation}
a+b+c=-3p 
\end{equation}
As $a=b$, 
\begin{equation}
 2b+c=-3p
\end{equation}

\begin{equation}
ab+bc+ca=3q
\end{equation}
As $a=b$, 
\begin{equation}
b^{2}+2bc=3q
\end{equation}

\begin{equation}
abc=-3pq
\end{equation}

As $a=b$, 

\begin{equation}
b^{2}c=-3pq
\end{equation}

Since $-3pq=\frac{(-3p)(3q)}{3}=\frac{(2b+c)(b^{2}+2bc)}{3}$ 
$\Rightarrow 3b^{2}c=2b^{3}+b^{2}c+4b^{2}c+2bc^{2}$ from equation (6), (2) and (4)\\
$\Rightarrow 0=2b^{3}+2b^{2}c+2bc^{2}$\\
\begin{equation}
b^{3}+b^{2}c+bc^{2}=0 
\end{equation}
But,$a^{2}b+b^{2}c+c^{2}a=b^{3}+b^{2}c+bc^{2}$ as $a=b$\\

Thus, $a^{2}b+b^{2}c+c^{2}a=b^{3}+b^{2}c+bc^{2}=0$.(from equation (7) )\\
 
\end{document}