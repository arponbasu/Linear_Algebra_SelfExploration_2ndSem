\documentclass[10pt,a4paper]{extarticle}
\usepackage[a4paper,margin=6mm]{geometry}
\usepackage{amsmath}
\usepackage{amssymb}

\begin{document}
\begin{center}


Arpon Basu, School: AECS-4 , Mumbai-400094

Solution for problem U401
\end{center}

We know that $p(x)=\frac{1}{x^2}$ for $x \in \{1,2,\dots , n+1\}$
we define $g(x)=x^2p(x)-1$. As degree of $g(x)$ is $(n+2)$ so it has $(n+2)$ roots.\\
$$g(x)=x^2p(x)-1= x^2\frac{1}{x^2} -1 =0 , \forall x\in \{1,2,\dots , n+1\}$$ 
$\therefore (n+1) $ roots of $g(x)$ are $ \{1,2,\dots , n+1\} $.
$$\therefore g(x)=(x-1)(x-2) \dots (x-(n+1))(x-\alpha) $$
$$\Rightarrow g(0)=(-1)^{n+1}(n+1)!(-\alpha) $$
$$\Rightarrow 0^2p(0)-1 =(-1)^{n+2}(n+1)!(\alpha) $$
$$\Rightarrow (-1)^{n+2}(n+1)!(\alpha)=-1 $$
$$ \alpha =\frac{(-1)^{n+1}}{(n+1)!} $$

$$ \therefore g(n+2)=\big((n+2)-1\big)\big((n+2)-2\big) \dots \big((n+2)-(n+1)\big)\Big((n+2)-\frac{(-1)^{n+1}}{(n+1)!}\Big) $$
$$ =(n+1)! \big (\frac{(n+2)!-(-1)^{n+1}}{(n+1)!} \big ) = (n+2)! + (-1)^{n+2} $$
$$ \therefore (n+2)^2 p(n+2) -1 =(n+2)! + (-1)^{n+2}  $$ 

$$ \therefore p(n+2)=\dfrac{(n+2)! + (-1)^{n+2}+1}{(n+2)^2}$$

Author wishes to thank his senior Adnan Ali for a fruitful conversation on the problem.


\end{document}


