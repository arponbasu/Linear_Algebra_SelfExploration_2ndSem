\documentclass[a4paper,10pt]{article}
\usepackage[utf8x]{inputenc}
\usepackage{amsmath}
\usepackage{amssymb}

\begin{document}
\begin{center}


Arpon Basu,School: AECS-4,Mumbai-400094

Solution for problem O435
\end{center}


We observe that if $a=\frac{x}{y+z}$ , $b=\frac{y}{z+x}$ , $c=\frac{z}{x+y}$ for some $x,y,z \in R_{\ge 0}$ , then the condition
$ab+bc+ca+2abc=1$ is satisfied.we also note that the mapping $(a,b,c) \mapsto (x,y,z)$ is bijective.\\
Thus

$$\sum_{cyclic} \frac{1}{8a^2 +1} = \sum_{cyclic} \frac{1}{8(\frac{x}{y+z})^2 +1}  = \sum_{cyclic} \frac{(y+z)^2}{8x^2 + (y+z)^2} $$

Since the above expression is homogeneous in $(x,y,z)$ , we put $x+y+z=1$ and consequently $x,y,z \in (0,1)$ to get 

$$E(x,y,z)= \sum_{cyclic} \frac{(1-x)^2}{8x^2 + (1-x)^2} =  \sum_{cyclic} \frac{x^2 -2x+1}{9x^2 -2x+1} $$

We thus ahve to minimize $E$ under $g: x+y+z-1=0$. Thus by Lagrange multipliers,

  $$\mathcal{L}(x,y,z,\lambda)=E(x,y,z)-\lambda g(x,y,z) $$



\begin{align*}
 &\nabla_{x,y,z,\lambda} \mathcal{L}=0 \\
 &\therefore \lambda = \frac{16x(x-1)}{(9x^2 -2x+1)^2}=\frac{16y(y-1)}{(9y^2 -2y+1)^2}=\frac{16z(z-1)}{(9z^2 -2x+1)^2}
\end{align*}

let 

\begin{align*}
 & f(r)=\frac{16r(r-1)}{(9r^2 -2r+1)^2} \\
 & \therefore f(x)=f(y)=f(z)=\lambda
\end{align*}

Now, $f^{'}(x)$ changes sign exactly one in $(0,1)$. Thus any line $y=\lambda $ can intersect $y=f(x)$ ( for $x\in(0,1)$) atmost twice. 
Hence $(x,y,z)$ can contain atmost two distinct numbers.\\
Thus the unordered triplet $(x,y,z)$ staisfying $f(x)=f(y)=f(z)$ is of the form $(\alpha,\alpha,1-2\alpha)$.

$$ \therefore f(\alpha)=f(1-2\alpha) $$

\begin{equation}
  \Rightarrow \frac{16\alpha(\alpha-1)}{(9\alpha^2 -2\alpha+1)^2}=\frac{16(1-2\alpha)(-2\alpha)}{(9(1-2\alpha)^2-2(1-2\alpha)+1)^2}
\end{equation}

  Solving equation 1 with Wolfram Alpha we get $\alpha = \frac{1}{3}, 0.474119$ (Note that $\alpha \in (0,1)$). Actually evaluating 
  $E(x,y,z) $ for $(\frac{1}{3},\frac{1}{3},\frac{1}{3})$ and $(\alpha_{0},\alpha_{0},1-2\alpha_{0})= 8.7033     >1$
  
  Thus $E(x,y,z)$ attains its global minima on the plane $x+y+z=1$ at  $(\frac{1}{3},\frac{1}{3},\frac{1}{3})$ at that point its value is $1$.
  
  $$\therefore E(x,y,z) \ge 1 $$
\begin{align*}
 & \therefore E(x,y,z) \ge 1 \\
 & \Rightarrow \frac{1}{8a^2+1} + \frac{1}{8b^2+1}+\frac{1}{8c^2+1} \ge 1 &\ \text{with equality} &\ a=b=c=\frac{1}{2}
\end{align*}

\end{document}
