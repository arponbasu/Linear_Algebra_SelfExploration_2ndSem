\documentclass{article}
\usepackage{amsmath}
\usepackage{amssymb}

\newcommand{\R}{\mathbb{R}}
\usepackage[utf8]{inputenc}

\title{\textbf{Dense divisors of a null matrix}}


\usepackage{natbib}
\usepackage{graphicx}

\begin{document}

\maketitle
\noindent
{\bf Statement:}\\ 
Prove that $\forall$ $A$ $\in$ $\mathbb{K}^{n\times n}$ $\exists$ $X$ $\in$ $\mathbb{K}^{n\times n}$ such that $AX$ = $\textbf{O}$ and $X$ has atmost \\
$nr$ zeroes in it, where $r$ = rank($A$).\\
A statement differing from the above only in the order of multiplication of the two matrices still holds, i.e, $\forall$ $A$ $\in$ $\mathbb{K}^{n\times n}$ $\exists$ $Y$ $\in$ $\mathbb{K}^{n\times n}$ such that $YA$ = $\textbf{O}$ and $Y$ has at most $n r$ zeroes in it. \\ \\
\noindent
{\bf Proof:}\\ 
Note that if $A$ is an invertible matrix, then $X = Y = \textbf{O}$ does the job for us.
Also, if $A$ = $\textbf{O}$, then any matrix $X$ $\in$ $\mathbb{K}^{n\times n}$ suffices, and once again the theorem is trivial (note that rank($\textbf{O}$) = 0).\\
If not, then for any $n\times n$ (non invertible) matrix $A$ with rank $r \geq 1$, we have $(n - r)$ ``basic" solutions of the equation $A\textbf{x} = \textbf{O}$, $\textbf{x} \in \mathbb{K}^{n\times 1}$, ie:- column vectors whose entries are 1 at the position of any one of the $(n - r)$ non-pivotal variable(s) and zero at the rest of the non-pivotal variables, while the pivotal variables are determined by back substitution. Thus, by taking a linear combination (with all coefficients non-zero) of all the the basic solutions of the matrix, one can generate a new column vector which is a solution of $A\textbf{x} = \textbf{O}$ and all it's non-pivotal entries are non-zero, i.e, it has at most $r$ zero entries in it. Stack that column vector together side by side, with itself, $n$ times to get a $n\times n$ matrix $X$ such that $AX$ = $\textbf{O}$ and $X$ has at most $n\cdot$ $r$ zeroes in it.\\
And as for the second part of the statement, we know $\exists$ $X'$ $\in$ $\mathbb{K}^{n\times n}$ such that $A^{T}X'$ = $\textbf{O}$ and $X'$ has at most $n r$ zeroes in it. Since rank$(A^{T})$ = rank$(A)$, and $(A^{T}X')^{T}$ = $(X')^{T}A$ = $\textbf{O}$, we get that $Y$ = $(X')^{T}$ satisfies all the desired properties.
Hence proved.\\ \\
\noindent



\bibliographystyle{plain}

\end{document}